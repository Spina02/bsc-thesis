\chapter*{Ringraziamenti}

\addcontentsline{toc}{chapter}{Acknowledgements}

Ed eccmi qua, in quel di Trieste a scrivere senza dubbio la parte più difficile della tesi, i ringraziamenti. Non sono mai stato brravo in queste cose, ma \textit{"qualche"} anno di scoutismo mi ha fatto un po' impratichire, quindi proverò a dare il meglio di me. 

Vorrei ringrazziare per prima la famiglia, che mi ha sempre supportato nelle mie scelte e sopportato per come sono. Mamma, Papà, avete sempre creduto in me e mi avete sempre sostenuto, anche quando io in primis non credevo in me stesso. Grazie inoltre per avermi dato la possibilità di continuare il mio percorso di studi qua a Trieste, non è un'esperienza che tutti possono permettersi e sono grato di aver avuto questa opportunità. Elly Folly, la mia sorellona, ti ringrazio per tutti i litigi e per tutte le risate che ci siamo fatti insieme, e per i preziosi consigli senza i quali probabilmente ancora non sarei iscritto alla Sapienza. Ringrazio Nonna per avermi permesso più di tutti di andare in uni, grazie al mitico bolide rosso che mi ha aiutato più volte a fare avanti e indietro.

Ringrazio Ale, cugino, ma più di tutti amico, grazie per essere sempre stato al mio fianco, per aver condiviso con me momenti di gioia e difficoltà, per tutte le passeggiate e le chiacchierate che ci siamo fatti insieme.

Ringrazio Fra, Dani, Fede, Sbob e Lello e tutti gli altri amici \textit{"di Torraccia"}, siete sempre stati presenti e quando la sera mi serviva una valvola di sfogo sapevo sempre che avrei trovato qualcuno pronto ad uscire.

Ringrazio i miei compagni dell'università, che col tempo siamo diventati un po' più che semplici \textit{"colleghi"}. Gigio, ti ringrazio per aver fatto quella battuta sul Conti a lezione di Analisi 1, da lì è nata una bellissima amicizia. Grazie per tutte le serate alle Freccie o a casa tua, per tutte le peroni offerte le bg, e le chiacchierate infinite fino alle 2 di notte sul tuo terrazzo.

    Paffo, il mio nerd preferito, grazie per aver condiviso con me la passione per la programmazione, per tutti i progetti iniziati e mai terminati, come il fantastico gioco in python che ha rischiato di farci perdere una sessione di esami. Un sincero grazie per avermi aiutato a revisionare la tesi 20 minuti prima della consegna mentre io scrivevo la conclusione. 

    Mattia, spesso sei stato il mio punto di riferimento quando brancolavo nel buio nella preparazione di alcune materie. Ti ringrazio per le infinite ore trascorse insieme al Laboratorio e per le infinite discussioni su argomenti per noi molto affascinanti.
    
    Cami, che anche se ultimamente ci siamo un po' allontanati, con la tua onniscenza sei sempre stata sempre pronta a rispondere a qualsiasi mio dubbio; non dimenticherò mai il super squadrone di studio con te e Kristi: durante le vacanze natalizie ho visto più voi che la mia famiglia, ma ciò è stato fondamentale per preparare analisi 1, che segnò l'inizio della mia carriera universitaria.

    Ale e Filo, voi siete stati fondamentali per farmi passare questi 3 anni con più leggerezza, grazie a voi l'università non è stata solo studio, ma anche tante risate e avventure.

    Ringrazio tutti gli altri amici dell'uni, che non ho citato, ma che incontravo ogni giorno a lezione, e che mi hanno sempre spronato a dare il meglio di me stesso.

Ringrazio Mirko, Claudia, Massi, Walter e Valerio: con voi ho condiviso momenti indimenticabili, vi ringrazio per tutte le uscite, i campi e le route fatte insieme. Grazie per tutti fantastici viaggi fatti assieme e per tutte le serate in cui ci sforzavamo invano a non parlare di scout.
Un grazie va anche a tutti gli altri miei amici di scout, che per molti anni sono stati per me come una famiglia (anche perchè passavo più tempo in sede che a casa). Con voi ho condiviso un percorso che è stato fondamentale per la mia crescita personale e che mi ha dato i mezzi per affrontare le sfide che la vita mi ha posto davanti.

Graie \textit{Poressio} per avermi fatto scoprire il mondo dell'informatica. Se non fosse stato per te probabilmente non avrei mai scelto di iscrivermi a questo corso. Grazie per la tua enorme pazienza e per avermi insegnato le basi della programmazione. Per me sei stato più volte un punto di rriferimento e un esempio da seguire.

Ringrazio \textit{McMirko}, che con il tuo umorismo e la tua simpatia hai alleggerito le ore più pesanti dietro ai banchi di scuola. Anche se non ci vediamo ormai molto spesso, rimani sempre un'amicizia costante nella mia vita.

Una menzione d'onore va al mitico co-relatore Germano Pasquale Tazio, che con la sua sapienza infinita mi ha aiutato a scrivere questa tesi in soli 10 giorni.

Infine ringrazio tutti coloro che non ho citato, ma che in tutti questi mi sono stati affianco, anche solo per un periodo di passaggio, e che mi hanno aiutato a crescere e a diventare la persona che sono oggi.