\chapter*{Ringraziamenti}

\addcontentsline{toc}{chapter}{Acknowledgements}

Ed eccomi qua, in quel di Trieste, a scrivere senza dubbio la parte più difficile della tesi: i ringraziamenti. Non sono mai stato bravo in queste cose, ma \textit{"qualche"} anno di scoutismo mi ha reso un po' più pratico, quindi proverò a dare il meglio di me.

Vorrei ringraziare innanzitutto la mia famiglia, che mi ha sempre supportato nelle mie scelte e sopportato per come sono. \textit{Mamma}, \textit{Papà}, avete sempre creduto in me e mi avete sostenuto, anche quando io in primis non credevo in me stesso. Vi sono grato, inoltre, per avermi dato la possibilità di continuare i miei studi qui a Trieste, una fantastica opportunità che non tutti possono permettersi. \textit{Elly Folly}, la mia sorellona, grazie per tutti i litigi e le risate, e per i preziosi consigli senza i quali probabilmente non sarei nemmeno iscritto alla Sapienza. Ringrazio \textit{Nonna}, perché più di tutti mi ha permesso di frequentare l’università, grazie anche al mitico bolide rosso che mi ha aiutato più volte a spostarmi avanti e indietro da casa.

Ringrazio \textit{Ale}, cugino ma soprattutto amico, per essere sempre stato al mio fianco, per aver condiviso con me momenti di gioia e difficoltà, per tutte le passeggiate e le chiacchierate fatte insieme. Ringrazio \textit{Fra}, \textit{Dani}, \textit{Fede}, \textit{Sbob}, \textit{Lello} e tutti gli amici \textit{"di Torraccia"}: siete sempre stati presenti. Quando la sera avevo bisogno di una valvola di sfogo, sapevo che avrei trovato qualcuno pronto a uscire.

Ringrazio i miei compagni dell'università, che col tempo sono diventati più che semplici \textit{"colleghi"}. \textit{Gigio}, grazie per quella battuta sul Conti a lezione di Analisi 1, da lì è nata una bellissima amicizia. Grazie per tutte le serate alle “Frecce” o a casa tua, per tutte le Peroni offerte, le bg e le chiacchierate infinite fino alle due di notte sul tuo terrazzo.\textit{Paffo}, il mio nerd preferito, grazie per aver condiviso con me la passione per la programmazione, per tutti i progetti iniziati e mai terminati, come quel fantastico gioco in Python che ha quasi rischiato di farci perdere una sessione d'esami. Un sincero grazie per avermi aiutato a revisionare la tesi venti minuti prima della consegna, mentre io ancora scrivevo le conclusioni. \textit{Mattia}, spesso sei stato il mio punto di riferimento quando brancolavo nel buio nella preparazione di alcune materie. Ti ringrazio per le infinite ore trascorse insieme in Laboratorio e per le discussioni su argomenti che per noi erano davvero affascinanti. \textit{Cami}, anche se ultimamente ci siamo un po' allontanati, con la tua onniscienza sei sempre stata pronta a rispondere a qualsiasi mio dubbio. Non dimenticherò mai il super “squadrone di studio” con te e \textit{Kristi}: durante le vacanze natalizie ho visto più voi che la mia famiglia, ma ciò è stato fondamentale per superare Analisi 1, l’inizio vero della mia carriera universitaria. \textit{Ale} e \textit{Filo}, voi siete stati fondamentali per trascorrere questi tre anni con più leggerezza: grazie a voi l'università non è stata solo studio, ma anche risate e avventure. Ringrazio tutti gli altri amici dell’università, che non ho citato, ma che incontravo ogni giorno a lezione e che mi hanno spronato a dare sempre il meglio di me stesso.

Ringrazio \textit{Mirko}, \textit{Claudia}, \textit{Massi}, \textit{Walter} e \textit{Valerio}: con voi ho condiviso momenti indimenticabili, vi ringrazio per tutte le uscite, i campi e le route fatte insieme. Grazie per tutti i fantastici viaggi e per le serate in cui ci sforzavamo, invano, di non parlare di scout. Un grazie va anche a tutti gli altri amici scout, che per molti anni sono stati per me come una seconda famiglia (anche perché passavo più tempo in sede che a casa). Con voi ho vissuto un percorso fondamentale per la mia crescita personale, che mi ha dato gli strumenti per affrontare le sfide della vita.

Grazie \textit{Poressio} per avermi fatto scoprire il mondo dell'informatica. Se non fosse stato per te, probabilmente non avrei mai scelto di iscrivermi a questo corso. Grazie per la tua enorme pazienza e per avermi insegnato le basi della programmazione. Per me sei stato più volte un punto di riferimento e un esempio da seguire.

Ringrazio \textit{McMirko}, che con il suo umorismo e la sua simpatia ha alleggerito le ore più pesanti dietro i banchi di scuola. Anche se ormai ci vediamo di rado, rimani una presenza costante nella mia vita.

Una menzione d'onore va al mitico co-relatore \textit{Germano Pasquale Tazio}, che con la sua infinita sapienza mi ha aiutato a scrivere questa tesi in soli dieci giorni.

Infine, ringrazio tutti coloro che non ho citato, ma che nel corso di questi anni mi sono stati accanto, anche solo per un breve periodo, aiutandomi a crescere e a diventare la persona che sono oggi.