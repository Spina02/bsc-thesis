\chapter{Valutazione dei risultati e metriche di qualità} \label{chap:evaluation}

In questo capitolo vengono presentate le metriche comunemente utilizzate in astrofotografia, e quelle utilizzate nell'ambito di questo progetto per valutare la qualità delle immagini ottenute attraverso il processo di elaborazione descritto nei capitoli precedenti.

La scelta delle metriche è fondamentale per avere un criterio oggettivo di valutazione e per confrontare i risultati ottenuti con diverse configurazioni di parametri. Verranno discusse sia metriche con riferimento, che richiedono un'immagine di riferimento ideale, sia metriche senza riferimento, che valutano la qualità dell'immagine in maniera autonoma.

A seguire verranno presentati i risultati ottenuti e verranno analizzati i miglioramenti introdotti dalle varie tecniche di elaborazione delle immagini.

\section{Metriche di valutazione con riferimento} \label{sec:r_metrics}

Le metriche con riferimento confrontano l'immagine elaborata con un'immagine ideale, fornendo una misura della similarità o della qualità relativa tra le due immagini. Nel contesto di questo progetto, l'utilizzo di metriche con riferimento ha presentato alcune sfide, principalmente a causa dell'assenza di un'immagine di riferimento adeguata, come discusso nella \cref{sec:challenges}.

\subsection{SSIM (Structural Similarity Index Measure)} \label{subsec:ssim}

\textbf{SSIM} è una metrica che misura la similarità strutturale tra due immagini. In astrofotografia, dopo aver applicato tecniche di denoising, la metrica SSIM viene utilizzata per valutare quanto efficacemente il rumore è stato ridotto mantenendo intatte le strutture originali dell'immagine, come stelle, galassie e nebulose. SSIM può essere utilizzata anche per perfezionare la tecnica dello stacking,aiutando a determinare quali combinazioni di immagini offrono la migliore preservazione delle strutture dettagliate rispetto alle immagini di riferimento. 

La formula generale per il calcolo dell'SSIM tra due immagini $x$ e $y$ è:

$$
\text{SSIM}(x, y) = \frac{(2\mu_x\mu_y + C_1)(2\sigma_{xy} + C_2)}{(\mu_x^2 + \mu_y^2 + C_1)(\sigma_x^2 + \sigma_y^2 + C_2)}
$$

dove $\mu_x$ e $\mu_y$ sono le medie delle immagini $x$ e $y$, $\sigma_x^2$ e $\sigma_y^2$ sono le varianze, $\sigma_{xy}$ è la covarianza tra $x$ e $y$ e, infine, $C_1$ e $C_2$ sono due costanti che stabilizzano la divisione in caso di valori molto bassi di $\mu_x^2 + \mu_y^2$ e $\sigma_x^2 + \sigma_y^2$. Il valore dell'SSIM è compreso tra $-1$ e $1$, dove $1$ indica una similarità perfetta tra le due immagini, e $-1$ indica una dissimilarità totale.

SSIM tiene conto della percezione umana, valutando la luminanza, il contrasto e la struttura delle immagini. Tuttavia, è sensibile a piccoli spostamenti e rotazioni tra le immagini, e non è in grado di valutare la qualità di immagini con diverse dimensioni o scale.

Nel progetto, a causa dell'assenza di un'immagine di riferimento adeguata (come discusso nella \cref{sec:challenges}), l'utilizzo dell'SSIM non è stato possibile.

\subsection{SNR (Signal-to-Noise Ratio)} \label{subsec:snr}

Il \textbf{Rapporto Segnale-Rumore} (SNR) è una metrica fondamentale nell'ambito dell'astrofotografia, utilizzata per quantificare la qualità delle immagini astronomiche. Esso misura la relazione tra il segnale utile (le informazioni desiderate, come stelle, galassie o, in questo caso, la luna e i suoi dettagli) e il rumore di fondo presente nell'immagine. Un alto valore di SNR indica che il segnale è dominante rispetto al rumore, risultando in immagini più nitide e dettagliate.

$$
SNR = 10 \cdot \log_{10} \left( \frac {\text{Potenza del Segnale}} {\text{Potenza del Rumore}} \right) = 10 \cdot \log_{10} \left( \frac{\mu_s^2}{\mu_n^2} \right)
$$

dove $\mu_s$ è la media dell'immagine del segnale e $\sigma_n^2$ è la varianza del rumore.

L'SNR è una metrica semplice e intuitiva, ma non tiene conto delle caratteristiche percettive del sistema visivo umano. Inoltre, come SSIM, è sensibile a piccoli spostamenti e rotazioni tra le immagini, e non è in grado di valutare la qualità di immagini con diverse dimensioni o scale, richiedendo dunque un perfetto allineamento tra le immagini di riferimento e quelle elaborate.

Algoritmi di denoising avanzati, inclusi quelli basati su reti neurali, vengono valutati utilizzando il SNR per garantire che il segnale sia preservato efficacemente mentre il rumore viene minimizzato. Tuttavia, anche l'SNR non è stato utilizzato in questo progetto a causa dell'assenza di un'immagine di riferimento adeguata.

\section{Metriche di valutazione senza riferimento} \label{sec:nr_metrics}

\subsection{NIQE (Naturalness Image Quality Evaluator)} \label{subsec:niqe}

\subsection{BRISQUE (Blind/Referenceless Image Spatial Quality Evaluator)} \label{subsec:brisque}

\subsection{LIQE (Language-Image Quality Evaluator)} \label{subsec:liqe}

\subsection{Motivazione della scelta di LIQE come metrica di riferimento} \label{subsec:why_liqe}

\section{Analisi e miglioramenti ottenuti} \label{sec:analysis}

\subsection{Effetti della calibrazione} \label{subsec:analysis_cal}

\subsection{Impatto del denoising} \label{subsec:analysis_den}

\subsection{Benefici dello stacking} \label{subsec:analysis_stack}

\subsection{Miglioramenti con sharpening e contrasto} \label{subsec:analisys_post}

\cleardoublepage