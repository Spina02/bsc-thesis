\chapter{Elaborazione delle immagini lunari} \label{chap:techniques}

Questo capitolo si propone di approfondire le tecniche di elaborazione delle immagini lunari, illustrando le nozioni teoriche alla base degli algoritmi implementati nel progetto. Ogni tecnica verrà descritta in dettaglio, partendo dalla calibrazione e dall'allineamento, passando per il pre-processing e lo stacking, per concludere con il post-processing. Quando necessario, verranno forniti pseudocodice e descrizioni dei processi matematici applicati alle immagini; l'implementazione sarà invece discussa nel capitolo \ref{chap:implementation}.

\section{Calibrazione delle immagini} \label{sec:calibration}

\subsection{Bias Frames} \label{subsec:bias}

\subsection{Dark Frames} \label{subsec:dark}

\subsection{Flat Frames} \label{subsec:flat}

\section{Allineamento delle immagini} \label{sec:alignment}

\subsection{Feature Detection e Matching ORB, SIFT e SURF} \label{subsec:feature_detectoion}

\subsection{Trasformazioni omografiche} \label{subsec:homography}

\section{Pre-processing delle immagini} \label{sec:preprocessing}

\subsection{Denoising tramite reti neurali: DnCnn} \label{subsec:denoising}

\subsection{Unsharp Masking e personalizzazione} \label{subsec:unsharp_mask}

\section{Stacking delle immagini} \label{sec:stacking}

\subsection{Principi e vantaggi dello stacking} \label{subsec:stacking_intro}

\subsection{Algoritmi di stacking} \label{subsec:atacking_algo}

\section{Post-Processing delle immagini} \label{sec:postprocess}

\subsection{Miglioramento di nitidezza e contrasto} \label{subsec:contrast}

\cleardoublepage