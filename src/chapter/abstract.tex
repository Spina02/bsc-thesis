\begin{abstract}
    Questa tesi presenta lo sviluppo di un software per l'elaborazione di immagini lunari, combinando tecniche tradizionali di image processing con approcci all'avanguardia basati su deep learning. Il progetto implementa un workflow completo e automatizzato che include calibrazione delle immagini raw, allineamento automatico tramite feature detection e matching, riduzione del rumore mediante DnCNN (Denoising Convolutional Neural Network), aumento della nitidezza tramite Unsharp Masking e diversi metodi di stacking per la creazione di immagini finali ad alta qualità.
    
    Particolare attenzione è stata dedicata allo sviluppo di un algoritmo di Unsharp Masking personalizzato che integra DnCNN con maschere basate sul gradiente dell'immagine. Questo approccio innovativo permette di preservare i dettagli significativi durante la riduzione del rumore, evitando la tipica perdita di informazioni che caratterizza i metodi tradizionali. Il sistema è stato testato su un dataset di immagini lunari acquisite con una fotocamera bridge Fujifilm FinePix S1, dimostrando come sia possibile ottenere risultati di qualità professionale anche con attrezzatura amatoriale, rendendo l'astrofotografia più accessibile.
    
    La valutazione quantitativa dei risultati è stata effettuata utilizzando multiple metriche di qualità dell'immagine senza riferimento, tra cui BRISQUE (Blind/Referenceless Image Spatial Quality Evaluator), NIQE (Natural Image Quality Evaluator) e LIQE (Learning-based Image Quality Evaluator). I risultati sperimentali hanno evidenziato miglioramenti significativi nelle diverse fasi di elaborazione, con particolare efficacia nella riduzione del rumore e nel recupero dei dettagli superficiali lunari. Il progetto fornisce una solida base per futuri sviluppi nell'ambito dell'elaborazione automatica di immagini astronomiche, con possibili applicazioni in altri contesti di imaging scientifico.
\end{abstract}

\let\cleardoublepage\clearpage
