\chapter*{Conclusions}

\addcontentsline{toc}{chapter}{Conclusions}

%! ----------------------------------------------- da integrare col resto -----------------------------------------------
Le metriche utilizzate, pur essendo senza riferimento, hanno mostrato una buona correlazione con la qualità percepita delle immagini. Tuttavia, alcune limitazioni sono emerse, in particolare nella capacità delle metriche di catturare miglioramenti specifici legati alle caratteristiche uniche delle immagini lunari.

Le tecniche di elaborazione implementate si sono dimostrate efficaci nel contesto del progetto. La combinazione di calibrazione, denoising avanzato, stacking ottimizzato e post-processing ha permesso di ottenere immagini di alta qualità, evidenziando dettagli altrimenti non visibili.
%! ----------------------------------------------------------------------------------------------------------------------
\textbf{Sviluppi futuri}

Il progetto ha portato allo sviluppo di una pipeline completa per l'elaborazione delle immagini lunari, ottenendo risultati soddisfacenti. Attraverso l'analisi del lavoro svolto e dei risultati ottenuti, sono state identificate diverse aree chiave che potrebbero beneficiare di ulteriori sviluppi e miglioramenti:

\begin{itemize}
    \item \textbf{Data augmentation}: come già specificato, l'implementazione di un sistema di data augmentation per generare immagini con rumore aggiunto a partire da un'immagine di riferimento di alta qualità potrebbe migliorare la qualità dei risultati e la precisione della pipeline.
    
    \item \textbf{Ottimizzazione dei parametri}: l'implementazione di un sistema di ricerca automatica dei parametri ottimali, basato su tecniche di ottimizzazione come l'ottimizzazione bayesiana o l'ottimizzazione genetica, potrebbe migliorare la qualità dei risultati e ridurre la dipendenza da parametri manuali.
    
    \item \textbf{Metriche di valutazione più avanzate}: l'implementazione di metriche di valutazione più avanzate, come l'indice SSIM (Structural Similarity Index) o l'indice PSNR (Peak Signal-to-Noise Ratio) potrebbe fornire una valutazione più accurata della qualità delle immagini, alternativamente si potrebbe pensare di addestrare un modello come LIQE o BRISQUE direttamente con immagini lunari, per ottenere una valutazione più precisa.
    
    \item \textbf{Interfaccia grafica (GUI)}: l'implementazione di un'interfaccia grafica per la pipeline potrebbe semplificare l'uso del programma. Ciò potrebbe includere funzionalità come la selezione delle immagini, la visualizzazione dei risultati e la regolazione dei parametri mostrando i loro effetti in tempo reale.
    
    \item \textbf{Parallelizzazione e ottimizzazione del codice}: l'implementazione di tecniche di parallelizzazione e ottimizzazione del codice potrebbe migliorare le prestazioni della pipeline e ridurre i tempi di esecuzione.
\end{itemize}

Implementando queste funzionalità, si potrebbe ottenere una pipeline più efficiente, precisa e facile da utilizzare, consentendo di ottenere risultati migliori e più affidabili nell'elaborazione di immagini astronomiche, oltre che aumentare la portabilità del programma, rendendolo più accessibile a utenti non esperti.

È importante notare che alcuni di questi miglioramenti, come l'implementazione di generatori per la lettura delle immagini, sono già in fase di sviluppo iniziale. Tuttavia, necessitano di ulteriore lavoro di implementazione, testing e ottimizzazione per essere integrati completamente nella pipeline operativa.

\cleardoublepage